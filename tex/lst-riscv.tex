% https://gist.github.com/AntonLydike/e339c3c3a4dcab8bc3c620b3fa436cda 
% RISC-V Assembler syntax and style for latex lstlisting package
% 
% These are risc-v commands as per our university (University Augsburg, Germany) guidelines.
%
% Author: Anton Lydike
%
% This code is in the public domain and free of licensing

% language definition
\lstdefinelanguage[RISC-V]{Assembler}
{
  alsoletter={.}, % allow dots in keywords
  alsodigit={0x}, % hex numbers are numbers too!
  morekeywords=[1]{ % instructions
    lb, lh, lw, lbu, lhu,
    sb, sh, sw,
    sll, slli, srl, srli, sra, srai,
    add, addi, sub, lui, auipc,
    xor, xori, or, ori, and, andi,
    slt, slti, sltu, sltiu,
    beq, bne, blt, bge, bltu, bgeu,
    j, jr, jal, jalr, ret,
    scall, break, nop
  },
  morekeywords=[2]{ % sections of our code and other directives
    .align, .ascii, .asciiz, .byte, .data, .double, .extern,
    .float, .globl, .half, .kdata, .ktext, .set, .space, .text, .word
  },
  morekeywords=[3]{ % registers
    zero, ra, sp, gp, tp, s0, fp,
    t0, t1, t2, t3, t4, t5, t6,
    s1, s2, s3, s4, s5, s6, s7, s8, s9, s10, s11,
    a0, a1, a2, a3, a4, a5, a6, a7,
    ft0, ft1, ft2, ft3, ft4, ft5, ft6, ft7,
    fs0, fs1, fs2, fs3, fs4, fs5, fs6, fs7, fs8, fs9, fs10, fs11,
    fa0, fa1, fa2, fa3, fa4, fa5, fa6, fa7
  },
  morecomment=[l]{;},   % mark ; as line comment start
  morecomment=[l]{//},   % mark ; as line comment start
  morecomment=[l]{\#},  % as well as # (even though it is unconventional)
  morestring=[b]",      % mark " as string start/end
  morestring=[b]'       % also mark ' as string start/end
}

% usage example:

% define some basic colors
\definecolor{mauve}{rgb}{0.58,0,0.82}

\lstdefinestyle{customrv}{
  % listings sonderzeichen (for german weirdness)
  literate={ö}{{\"o}}1
           {ä}{{\"a}}1
           {ü}{{\"u}}1,
  basicstyle=\tiny\ttfamily,                    % very small code
  breaklines=true,                              % break long lines
  commentstyle=\itshape\color{green!50!black},  % comments are green
  keywordstyle=[1]\color{blue!80!black},        % instructions are blue
  keywordstyle=[2]\color{orange!80!black},      % sections/other directives are orange
  keywordstyle=[3]\color{red!50!black},         % registers are red
  stringstyle=\color{mauve},                    % strings are from the telekom
  %identifierstyle=\color{teal},                 % user declared addresses are teal
  frame=l,                                      % black line on the left side of code
  language=[RISC-V]Assembler,                   % all code is RISC-V
  tabsize=4,                                    % indent tabs with 4 spaces
  showstringspaces=false                        % do not replace spaces with weird underlines
}